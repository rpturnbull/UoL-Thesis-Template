%!TEX root = ../thesis.tex
%*******************************************************************************
%*********************************** Second Chapter *****************************
%*******************************************************************************

\chapter{Literature Review} \label{Ch:LitRev}

%********************************** Introduction *******************************
\textit{Generally nice to have a chapter overview before the introduction. Here you talk about the chapter and how it fits in the thesis rather than your subject.}

\section{Introduction}
 90\textdegree 
 
 nice big sideways table
\begin{sidewaystable}
	\caption[Shoulder, clavicle, elbow, wrist, hand, spine, hip, knee, ankle and foot Degrees of Freedom and Range of Motion. * Where a DOF exists due to joint structure, but the movement has not quantified been quantified in the literature]{Shoulder, clavicle, elbow, wrist, hand, spine, hip, knee, ankle and foot Degrees of Freedom and Range of Motion. * Where a DOF exists due to joint structure, but the movement has not quantified been quantified in the literature}
	\centering
	\label{table:body_nat_dof}
	\resizebox{\textwidth}{!}{
		\begin{tabular}{l c c c c c c c c c c c}
			\toprule
			%\multirow{2}{*}{Dental measurement} & \multicolumn{2}{c}{Species I} & \multicolumn{2}{c}{Species II} &  \\
			Motion                               & Shoulder    & Clavicle        & Elbow       & Wrist            & Hand                  & Thoracic Spine        & Lumbar Spine          & Hip         & Knee                  & Ankle       & Foot             \\ \midrule
			Flexion                              & 45\textdegree         & \textendash     & 145\textdegree        & 80\textdegree              & \textendash           & 20-43\textdegree                & 40-60\textdegree                & 120\textdegree        & 120\textdegree                  & \textendash & \textendash      \\
			Extension                            & 180\textdegree        & \textendash     & 0-5\textdegree        & 70\textdegree              & \textendash           & \textendash           & 20-35\textdegree                & 10\textdegree         & 5\textdegree                    & \textendash & \textendash      \\
			Abduction                            & 50\textdegree         & \textendash     & \textendash & 41\textdegree              & \textasteriskcentered & \textendash           & \textendash           & 70\textdegree         & \textendash           & \textendash & \textendash      \\
			Adduction                            & 180\textdegree        & \textendash     & \textendash & 30\textdegree              & \textasteriskcentered & \textendash           & \textendash           & 70\textdegree         & \textendash           & \textendash & \textendash      \\
			Lateral Flexion                      & \textendash & \textendash     & \textendash & \textendash      & \textendash           & 6-30\textdegree                 & 16-18\textdegree                & \textendash & \textendash           & \textendash & \textendash      \\
			Lateral Extension                    & \textendash & \textendash     & \textendash & \textendash      & \textendash           & 5-31\textdegree                 & 18-21\textdegree                & \textendash & \textendash           & \textendash & \textendash      \\
			Lateral Rotation (Inversion)         & 80\textdegree         & \textendash     & \textendash & \textendash      & \textendash           & 20-43\textdegree                & 15-18\textdegree                & 50\textdegree         & 25\textdegree                   & \textendash & (12\textdegree)            \\
			Medial Rotation (Eversion)           & 90\textdegree         & \textendash     & \textendash & \textendash      & \textendash           & 29-56\textdegree                & 16-19\textdegree                & 50\textdegree         & 25\textdegree                   & \textendash & (23\textdegree)            \\
			Pronation                            & \textendash & \textendash     & \textendash & -90\textdegree             & \textendash           & \textendash           & \textendash           & \textendash & \textendash           & \multicolumn{2}{c}{Multi. DOF} \\
			Supination                           & \textendash & \textendash     & \textendash & 85\textdegree              & \textendash           & \textendash           & \textendash           & \textendash & \textendash           & \multicolumn{2}{c}{Multi. DOF} \\
			Retraction (Posterior Translation)   & \textendash & 12\textdegree             & \textendash & \textendash      & \textendash           & \textasteriskcentered & \textasteriskcentered & \textendash & \textasteriskcentered & \textendash & \textendash      \\
			Protraction (Anterior Translation)   & \textendash & 40\textdegree             & \textendash & \textendash      & \textendash           & \textasteriskcentered & \textasteriskcentered & \textendash & \textasteriskcentered & \textendash & \textendash      \\
			Elevation (Superior Translation)     & \textendash & 60\textdegree             & \textendash & \textendash      & \textendash           & \textasteriskcentered & \textasteriskcentered & \textendash & \textasteriskcentered & \textendash & \textendash      \\
			Depression (Inferior Translation)    & \textendash & 3\textdegree              & \textendash & \textendash      & \textendash           & \textasteriskcentered & \textasteriskcentered & \textendash & \textasteriskcentered & \textendash & \textendash      \\
			Plantarflexion                       & \textendash & \textendash     & \textendash & \textendash      & 65-110\textdegree               & \textendash           & \textendash           & \textendash & \textendash           & 40-55\textdegree      & \textendash      \\
			Dorsiflection                        & \textendash & \textendash     & \textendash & \textendash      & 0\textdegree                    & \textendash           & \textendash           & \textendash & \textendash           & 10-20\textdegree      & \textendash      \\
			Medial Translation                   & \textendash & \textendash     & \textendash & \textendash      & \textendash           & \textasteriskcentered & \textasteriskcentered & \textendash & \textasteriskcentered & \textendash & \textendash      \\
			Lateral Translation                  & \textendash & \textendash     & \textendash & \textendash      & \textendash           & \textasteriskcentered & \textasteriskcentered & \textendash & \textasteriskcentered & \textendash & \textendash      \\
			Valgus Rotation                      & \textendash & \textendash     & \textendash & \textendash      & \textendash           & \textendash           & \textendash           & \textendash & \textendash           & \textendash & \textendash      \\
			Varus Rotation                       & \textendash & \textendash     & \textendash & \textendash      & \textendash           & \textendash           & \textendash           & \textendash & \textendash           & \textendash & \textendash      \\
			Anterior Rotation                    & \textendash & 3\textdegree              & \textendash & \textendash      & \textendash           & \textendash           & \textendash           & \textendash & \textendash           & \textendash & \textendash      \\
			Posterior Rotation                   & \textendash & 30-50\textdegree          & \textendash & \textendash      & \textendash           & \textendash           & \textendash           & \textendash & \textendash           & \textendash & \textendash      \\
			Natural DOF                          & 3           & 2               & 1           & 3                & 20                    & 72                    & 30                    & 3           & 6                     & 1           & 1                \\
			Per Joint Unit                       &             &                 &             &                  & 2                     & 6                     & 6                     &             &                       &             &                  \\ \bottomrule
		\end{tabular}}
\end{sidewaystable}


\begin{itemize}
	\item \textbf{Body -} The simulations bones and links. They are defined by relating the position of a joint relative to a parent and child body. Where the body being inserted is the child body.
	\item \textbf{Joint -} Used to connect two bodies. A joint is specified when inserting a new body. Joint types selection depends on the desired functionality.
	\item \textbf{Geometry -} Used as interaction objects or simply providing body visualisation. 
\end{itemize}

\nomenclature[z-moco]{$MOCO$}{OpenSim MOtion and COntrol optimiser}
\nomenclature[z-scone]{$SCONE$}{Simulated Controller OptimizatioN Environment}

This is what a table looks like, easiest to build them as you want in excel then put the \& symbols in. Refere to a table like this: Table \ref{table:HRI_Force}

\begin{table}
	\caption[Summary of pressure experienced at the physical human-robot interface, including comfort levels and capillary compression pressure.]{Summary of pressure experienced at the physical human-robot interface, including comfort levels and capillary compression pressure.}
	\centering
	\label{table:HRI_Force}
	\resizebox{\textwidth}{!}{
	\begin{tabular}{c c c c c c}
		\toprule
		     \multirow{2}{*}{Reference}       &          \multicolumn{4}{c}{Pressure [kPa]}           & \multirow{2}{*}{Conditions}             \\
		                                      & Comfortable & Discomfort  & Damage      & Pain        &                                         \\ \midrule
		Schiele et al. \cite{Schiele2008afut} & 1.3-4       & >4          & \textendash & \textendash & While completing tasks                  \\ % 10-30mmHg
		Kermavnar et al. \cite{Kermavnar2020} & <14.1       & 14.1-27.5   & \textendash & 43.4-60.3   & Dynamic loading over short periods      \\ % Kermavnar2020 discomfort 14.1 - 27.5 kPa pain 43.4 - 60.3 kPa - 21 participants
		     Ryan et al. \cite{Ryan1989}      & \textendash & \textendash & 4-4.7       & \textendash & Capillaries occlusion  - minimum stress \\
		      Cho et al. \cite{Cho2012}       & \textendash & \textendash & 9           & \textendash & Injury possible from                    \\ % Ryan et al. 1989 - from Cho 2012 Min stress for harm 4-4.7kPa almost 9kPa for dermalogical damage
		    Agam et al. \cite{Agam2008a}      & \textendash & \textendash & >6.27       & \textendash & Average capillary pressure              \\ \bottomrule
	\end{tabular}}
\end{table}   



Maths equations (Equation \ref{eq:adjMod})


\begin{equation}
	\label{eq:adjMod}
	E_{i}^{*} = \frac{E_{i}}{1-\nu_{i}^{2}}
\end{equation}

Where $E_{i}^{*}$ = composite elastic modulus for a given material, $E_{i}$ = the elastic modulus for a given material and $\nu$ = material Poisson's ratio.

Maths formatting is done between dollar signs