%!TEX root = ../thesis.tex
%*******************************************************************************
%*********************************** First Chapter *****************************
%*******************************************************************************

\chapter{Introduction}


%*****************************Definition of Exoskeletons************************
\section{Background}\label{ref:background}
%What's your research
%Why are you doing this
%what have others done before you
%how and what do you add to it
%How will all of it benefit the field
%Point the reader towards the conclusion 

%Explains thesis to someone who won't read the whole thing

First off explain stuff like Human Robot Interaction (HRI) then add to nomenclature in the code and it will show up when you compile

\nomenclature[z-hri]{$HRI$}{Human Robot Interaction}

Next make a list of things. This can be bullet points or numbers

\begin{itemize}
	\item[] Things
	\item[] Stuffs
	\item[] More things
\end{itemize}



\begin{landscape}
	\begin{figure} 
		\centering    
		\includegraphics[width=1.2\textwidth]{Chapter1/Figs/dont-panic.png}
		\caption[This bit of text goes at the list of figures so might be slightly different.]{Example figure but landscape.}
		\label{fig:exoflow}
	\end{figure}
\end{landscape}


\begin{figure}
	\centering    
	\subfigure[]{\includegraphics[width=0.2\textwidth]{Chapter1/Figs/dont-panic.png}\label{fig:FractureProneA}}
	\subfigure[]{\includegraphics[width=0.2\textwidth]{Chapter1/Figs/dont-panic.png}\label{fig:FractureProneB}}
	\subfigure[]{\includegraphics[width=0.2\textwidth]{Chapter1/Figs/dont-panic.png}\label{fig:FractureProneC}}
	\caption[Subfigure example (a) don't panic, (b) panik (c) and another thing.]{Subfigure example (a) don't panic, (b) panik (c) and another thing.}
	\label{fig:FractureProne} % this last label needs to be here for subfigure lables to work (it can be empty).
\end{figure}

Things and stuff are written but now I want a footnote\footnote{Look a footnote, how exciting}


Cite a person using bibtex, you can get menedeley to export your whole library \cite{Cho2012} 



\section{Motivation for Research} \label{Attachment} % label if you want to cross reference section else where


\section{Aims and Objectives}

nice numerical list

\begin{enumerate}
	\item Where is the optimal contact position for the attachment of an assistive robotic exoskeleton to the limbs of the end user?
	\item How does the attachment location contribute to exoskeleton-human joint alignment?
	\item What is the relationship between attachment position and user metabolic expenditure?
\end{enumerate}

\subsection{Aims}

\subsection{Objectives}

\begin{enumerate}
	% Lower body 
	\item Things
	
	\item Stuff
	
	% \item more

\end{enumerate}



\section{Project Scope}

 \textit{italic text}  
 
\begin{figure} [ht]
	\centering    
	\includegraphics[width=120mm]{Chapter1/Figs/subfolder/dont-panic.png}
	\caption[Project scope flow diagram with areas of interest highlighted in red]{Project scope flow diagram with areas of interest highlighted in red}
	\label{fig:Scope}
\end{figure}



\section{Contributions of this Research}
\begin{enumerate}
	\item things
	\item stuff
	\item more
\end{enumerate}


\section{Thesis Organisation}

sometimes paragraph look better to break up formatting

\paragraph{Chapter 2: Literature Review -} 

\paragraph{Chapter 3: Name -}

\paragraph{Chapter 4: Name -} 

\paragraph{Chapter 5: Name -}

\paragraph{Chapter N: Summary, Conclusions and Future Work -}  This chapter summarises the work conducted during this research. Discusses simulations result implications and how they compare to the preliminary work and literature. The conclusions made from this work and finally recommendations for the further development of the work are set out.


