%!TEX root = ../thesis.tex
%*******************************************************************************
%*********************************** First Chapter *****************************
%*******************************************************************************

\chapter{Introduction}


\section{Background}\label{ref:background}
What's your research//
Why are you doing this//
what have others done before you//
how and what do you add to it//
How will all of it benefit the field//
Point the reader towards the conclusion //

Explains thesis to someone who won't read the whole thing//

First off explain stuff like Human Robot Interaction (HRI) then add to nomenclature in the code and it will show up when you compile

\nomenclature[z-hri]{$HRI$}{Human Robot Interaction}

Next make a list of things. This can be bullet points or numbers

\begin{itemize}
	\item[] Things
	\item[] Stuffs
	\item[] More things
\end{itemize}

\section{Example Figures}

you can either set the width mannually (width=120mm) or you can set it relative to textwidth (width=1.2/textwidth)

\begin{figure} [ht]
	\centering    
	\includegraphics[width=120mm]{Chapter1/Figs/subfolder/dont-panic.png}
	\caption[Regular figure]{Regular figure}
	\label{fig:Scope}
\end{figure}

\begin{landscape}
	\begin{figure} 
		\centering    
		\includegraphics[width=1.2\textwidth]{Chapter1/Figs/dont-panic.png}
		\caption[This bit of text goes in the list of figures so might be slightly different.]{Example figure but landscape.}
		\label{fig:exoflow}
	\end{figure}
\end{landscape}


\begin{figure}
	\centering    
	\subfigure[]{\includegraphics[width=0.2\textwidth]{Chapter1/Figs/dont-panic.png}\label{fig:FractureProneA}}
	\subfigure[]{\includegraphics[width=0.2\textwidth]{Chapter1/Figs/dont-panic.png}\label{fig:FractureProneB}}
	\subfigure[]{\includegraphics[width=0.2\textwidth]{Chapter1/Figs/dont-panic.png}\label{fig:FractureProneC}}
	\caption[Subfigure example (a) don't panic, (b) panik (c) and another thing.]{Subfigure example (a) don't panic, (b) panik (c) and another thing.}
	\label{fig:FractureProne} % this last label needs to be here for subfigure lables to work (it can be empty).
\end{figure}

Things and stuff are written but now I want a footnote\footnote{Look a footnote, how exciting}


Cite a person using bibtex, you can get menedeley to export your whole library \cite{Cho2012} 



\section{Motivation for Research} \label{Attachment} 

Use a label (in code) if you want to cross reference section elsewhere


\section{Aims and Objectives}


\subsection{Aims}

nice numerical list

\begin{enumerate}
	\item Highlight figure examples?
	\item Highlight citations, cross referencing and lists?
\end{enumerate}

\subsection{Objectives}

\begin{enumerate}
	\item Things
	\item Stuff
\end{enumerate}



\section{Project Scope}

 \textit{italic text}  and now in \textb{BOLD}
 




\section{Contributions of this Research}
\begin{enumerate}
	\item things
	\item stuff
	\item more
\end{enumerate}


\section{Thesis Organisation}

sometimes paragraph look better to break up formatting

\paragraph{Chapter 2: Literature Review -} I read some things about what other people did. Here is what I think about it.

\paragraph{Chapter 3: Method -} I thought about it and decided this was how I would tackle the problem.

\paragraph{Chapter 4: Preliminary Results -} I did some work to inform the design. It wasn't very good. 

\paragraph{Chapter 5: Design -} I built a thing and reviewed it.

\paragraph{Chapter 5: Results -} Used used the thing I built to test the thing I was interested in. This is how it went.

\paragraph{Chapter N: Summary, Conclusions and Future Work -}  This chapter summarises the work conducted during this research. Discusses simulations result implications and how they compare to the preliminary work and literature. The conclusions made from this work and finally recommendations for the further development of the work are set out.


